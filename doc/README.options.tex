\documentclass[11pt, a4paper ,twoside]{article}
\usepackage{epstopdf}
\usepackage{natbib}
\usepackage{epstopdf}
\usepackage[utf8]{inputenc}
\usepackage{a4wide}
\usepackage{rotating}
\usepackage{multicol}
\usepackage{afterpage}
\usepackage{color}
\usepackage{array}
\usepackage{floatflt}
\usepackage[colorlinks=true, linkcolor=blue, bookmarks=true, citecolor=blue, pagebackref]{hyperref}

\setcounter{secnumdepth}{3}
\setcounter{tocdepth}{1}
\usepackage[french]{minitoc}
\setcounter{parttocdepth}{2}
\usepackage{amsmath}

\usepackage[final]{pdfpages}
\usepackage{amssymb}
\usepackage[english, french]{babel}
\pagestyle{plain}

\begin{document}
There are two kinds of options for each module:
\begin{itemize}
\item inclusive (check boxes),
\item exclusives (radio boxes), only one choice per section.
\end{itemize}

For each of these options, there can be a text box that allows to
specify a value for that option, and even provide a default value.
The only condition for this is to append ``:'' to the name of that
option.

There is for instance: \verb|--marcel-stacksize:64| (64 being the
default value).

The following files are needed in \verb|module-name/config/options| to
define an option:
\begin{enumerate}
\item \verb|nnnoption-name:.sh|
\item \verb|nnnoption-name:.help| (to provide help on the option)
\item \verb|nnnoption-name:.dep| (for eventual dependencies)
\item \verb|nnnoption-name:.dft| (to specify a default value)
\end{enumerate}

\verb+nnn+ is the number of the option and is used for sorting them. It
should start with a 0 for inclusive options, and 1 for exclusive
options.

After adding/supressing/modifying options, \verb+pm2-recreate-links+ has
to be called to take them into account.

To configure flavors, either use \verb+ezflavor+, \verb+make config+, or
use the following commands:
\begin{itemize}
\item \verb|pm2-flavor get --flavor=flavor-name| : Shows the
flavor.
\item \verb|pm2-flavor list| : Give the list of flavors.
\item \verb|pm2-flavor set --flavor=flavor-name| : Reset the whole
flavor.
\item \verb|pm2-flavor set --flavor=flavor-name `cat FILE`| : Set
flavor options from a file.
\item \verb|xargs pm2-flavor set --flavor=flavor-name < FILE| : Other
way to achieve the same.
\item \verb|pm2-create-sample-flavor flavor-name| : Generate one of the
sample flavors.
\end{itemize}
\end{document}
