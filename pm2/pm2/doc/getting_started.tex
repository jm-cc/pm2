\documentclass[a4paper,11pt]{article}

% packages
\usepackage{moreverb}
\usepackage{fullpage}
\usepackage{boxedminipage}
\usepackage[dvips]{graphics}
\usepackage{url}
\usepackage{psboxit}
\usepackage{xspace}

% settings
\let\bfseriesaux=\bfseries \def\bfseries{\sffamily\bfseriesaux}


% macros
\def\pm2{PM$^2$\xspace}
\def\mad{\emph{Madeleine}\xspace}
\def\madii{\emph{Madeleine~II}\xspace}
\def\mar{\emph{Marcel}\xspace}


% first page
\title{Getting Started with \pm2}
\author{Gabriel Antoniu \and%
        Olivier Aumage \and%
        Luc Boug\'e \and%
        Jean-Fran\c{c}ois M\'ehaut \and%
        Raymond Namyst}


%%%%%%%%%%%%%%%%
\begin{document}
\maketitle


% Introduction
%______________
\section{Introduction : What is PM2?}

%===========
\subsection{Multithreaded Programming Environment}
        API (C language) + set of libraries

%===========
\subsection{Programing model}


% Installation
%_____________________________________________________________________________
\section{Installation}

%===========
\subsection{Currently Supported Platforms}

\subsubsection{OS/Processors}
\pm2 has been currently tested on the following platforms:
\begin{itemize}
\footnotesize
\item \textsc{Linux}: \textsc{x86}, \textsc{PPC}, \textsc{Alpha}
\item \textsc{Solaris}: \textsc{Sparc}, \textsc{x86}
\item \textsc{FreeBSD}: \textsc{x86}
\item \textsc{Aix}: \textsc{RS6000}
\item \textsc{Irix}: \textsc{Mips}
\item \textsc{OSF}: \textsc{Alpha}
\item \textsc{Unicos}: \textsc{Alpha}
\end{itemize}

\subsubsection{Network procotols}
The network subsystem of the \pm2 environment currently supports
the following protocols:
\begin{itemize}
\footnotesize
\item TCP
\item MPI (LAM-MPI, MPI-BIP)
\item VIA
\item SCI/SISCI
\item SBP
\end{itemize}

\subsubsection{Required development tools}
\pm2 relies on the availability of the following development tools:
\begin{itemize}
\footnotesize
\item GNU C Compiler \textsc{gcc} (versions 2.7.2.3, 2.8.1 and \textsc{Egcs}
versions have been successfully tested)
\item GNU Make
\end{itemize}
 
%===========
\subsection{Getting the PM2 software and help}
The \pm2 archive is available on the \pm2 web site at 
\url{http://www.ens-lyon.fr/~rnamyst/pm2.html}.

Contact guru: \url{bouge@ens-lyon.fr}

%===========
\subsection{Unpacking the PM2 distribution}
        $\rightarrow$ making a personal installation

\subsubsection{Contents of the package}
The \pm2 distribution is organized as follows:
\begin{footnotesize}
\begin{verbatim}
        pm2 --+
              |
              +-- bin........: PM2 scripts
              |  
              +-- doc........: documentation files
              |
              +-- dsm........: DSM-PM2 distributed shared memory manager
              |
              +-- examples...: variety of PM2 examples
              |
              +-- include....: PM2 header files
              |
              +-- lib........: PM2 library destination
              |
              +-- madeleine..: network subsystem
              |
              +-- make.......: PM2 makefiles
              |
              +-- marcel.....: multithreading management subsystem
              |
              +-- source.....: PM2 source files
              |
              +-- toolbox....: general purpose library
              
\end{verbatim}
\end{footnotesize}
\subsubsection{Environment variables}
        PM2\_ROOT, etc.
        PATH

%===========
\subsection{Configuring}

\subsubsection{OS/Processor}

\subsubsection{Network}
        pm2custom

\subsubsection{Advanced configuration}
        Makefiles $\rightarrow$ cf Advanced settings/Makefiles

%===========
\subsection{Compiling}
        not really required, but can be done by typing "cd pm2; make"

%===========
\subsection{Cleanning the directory tree}
        pm2clean


% PM2 Step by Step
%_____________________________________________________________________________
\section{PM2 step by step}

%===========
\subsection{The minimal PM2 program}
\begin{figure}
\begin{center}
\begin{boxedminipage}{0.8\textwidth}
\begin{footnotesize}
\begin{listing}{0}
 #include <stdio.h>
 #include <pm2.h>

 int pm2_main(int argc, char **argv)
 {
   pm2_init(&argc, argv);

   printf("Hello World !\n");
  
   if (pm2_self() == 0)
     pm2_halt();
  
   pm2_exit();
   return 0;
 }
\end{listing}
\end{footnotesize}
\end{boxedminipage}
\end{center}
\caption{Minimal \pm2 program\label{fig:ex1}}
\end{figure}
Let us start our \pm2 tour by writting the minimal \pm2 program. The
Figure~\ref{fig:ex1} shows an example of such a minimalist \pm2
code. As we can see, a \pm2 program must first include the \texttt{pm2.h}
header file along with other standards header files. 

We now write the program body. The line~7 illustrates that the main
function should be called \texttt{pm2\_main} instead of the classical
\texttt{main} when using \pm2. Yet, the arguments remain the regular
\texttt{argc}/\texttt{argv} pair. A call to \texttt{pm2\_init}
effectively initializes the \pm2 runtime system. Once the
\texttt{pm2\_init} function returns, the actual application specific
\emph{parallel processing} can begin. Here, this processing reduces to
a call to the \texttt{printf} function of the standard library. This
call will be concurrently performed on any process constituting the
session configuration during the program execution.

Finally, the termination phasis concludes our minimal \pm2 program. It
is splitted into two steps. One (and only one) of the nodes should
first perform a call to \texttt{pm2\_halt}. This step ensures that no
further incoming service request will be accepted by any of the
session processes. The \texttt{pm2\_self} function is used to retrieve
the local process rank. So, the \texttt{pm2\_halt} call is executed by
the process \texttt{0} (sometimes called the \emph{master} process).
Once each process reaches the \texttt{pm2\_exit} second termination
step, the \pm2 session ends.

%===========
\subsection{Running a PM2 program}
        $\rightarrow$ pm2conf, pm2load, pm2logs

%===========
\subsection{Common pitfalls}

%===========
\subsection{Getting the output in the console}
\begin{figure}
\begin{center}
\begin{boxedminipage}{0.8\textwidth}
\begin{footnotesize}
\begin{listing}{0}
 #include <stdio.h>
 #include <pm2.h>

 int pm2_main(int argc, char **argv)
 {
   pm2_init(&argc, argv);

   pm2_printf("Hello World !\n");
  
   if (pm2_self() == 0)
     pm2_halt();
  
   pm2_exit();
   return 0;
 }
\end{listing}
\end{footnotesize}
\end{boxedminipage}
\end{center}
\caption{Output redirection\label{fig:ex2}}
\end{figure}

%===========
\subsection{Invoking a remote service}

\subsubsection{Example}
\begin{figure}
\begin{center}
\begin{boxedminipage}{0.8\textwidth}
\begin{footnotesize}
\begin{listing}{0}
 #include <stdio.h>
 #include <pm2.h>

 static unsigned int service_id;

 static void service(void)
 {
   pm2_rawrpc_waitdata();
   printf("Hello, World!\n");
 }

 int pm2_main(int argc, char **argv)
 {
   pm2_rawrpc_register(&service_id, service)
   pm2_init(&argc, argv);

   if (pm2_self() == 0)
     {
       pm2_rawrpc_begin(1, service_id, NULL);
       pm2_rawrpc_end();

       pm2_halt();
     }
   pm2_exit();
   return 0;
 }
\end{listing}
\end{footnotesize}
\end{boxedminipage}
\end{center}
\caption{Sample service\label{fig:ex3}}
\end{figure}

%===========
\subsection{Passing parameters}

\subsubsection{Example}
\begin{figure}
\begin{center}
\begin{boxedminipage}{0.8\textwidth}
\begin{footnotesize}
\begin{listing}{0}
#include <stdio.h>
#include <pm2.h>

static unsigned int service_id;

static void service(void)
{
  int len;
  char *s;

  mad_unpack_int(mad_send_CHEAPER, mad_receive_EXPRESS, &len, 1);
  s = malloc(len);
  mad_unpack_char(mad_send_CHEAPER, mad_receive_CHEAPER, s, len);
  pm2_rawrpc_waitdata();

  printf("The sentence is %s\n", s);
}

int pm2_main(int argc, char **argv)
{
  pm2_rawrpc_register(&service_id, service)
  pm2_init(&argc, argv);
  if (pm2_self() == 0)
    {
      char s[] = "A la recherche du temps perdu.";
      int len;

      len = strlen(s) + 1;
      pm2_rawrpc_begin(1, service_id, NULL);
      mad_pack_int(mad_send_CHEAPER, mad_receive_EXPRESS, &len, 1);
      mad_pack_char(mad_send_CHEAPER, mad_receive_CHEAPER, s, len);
      pm2_rawrpc_end();

      pm2_halt();
    }
  pm2_exit();
  return 0;
}
\end{listing}
\end{footnotesize}
\end{boxedminipage}
\end{center}
\caption{Service with a string parameter\label{fig:ex4}}
\end{figure}

\subsubsection{Packing/unpacking API}
        pack\_int/pack\_byte
        send\_CHEAPER, etc.

\subsubsection{A more interesting example}

%===========
\subsection{"Treaded" services}
        Services handlers are executed sequentially
        Thread creation: pm2\_thread\_create
        Caveat! pm2\_waitdata...


% Advandced Threading
%_____________________________________________________________________________
\section{Advanced thread features}

%===========
\subsection{Thread synchronization}

%===========
\subsection{Invoking services synchronously}
        2 one-way invocations + synchronization

\subsubsection{Examples}
        Better: using the "pm2\_rawrpc\_wait" features

%===========
\subsection{Thread migration}
        Thread state : migratable/non-...

\subsubsection{Migration in 3 steps}
        freezing the local scheduler
        getting the list of migratable threads
        selecting and migrating threads

\subsubsection{Example}
        Iso-address migration

%===========
\subsection{Migrating with dynamically allocated data}

\subsubsection{Isomalloc}


% DSM
%_____________________________________________________________________________
\section{Sharing data across clusters}
        DSM-PM2


% FAQ
%_____________________________________________________________________________
\section{FAQ}
        Why does my program crash the entire cluster?
        Why do I get the message "Permission denied" when running pm2load?


% Bibliography
%_____________________________________________________________________________
\section{Bibliography}
                
\end{document}
_______________________________________________________________________________
$Log: getting_started.tex,v $
Revision 1.6  2000/01/14 17:16:00  oaumage
- ajout des exemples 2, 3 et 4

Revision 1.5  2000/01/14 16:06:15  oaumage
- exemple minimal + texte

Revision 1.4  2000/01/14 15:38:05  oaumage
- nouveau programme minimal

Revision 1.3  2000/01/14 14:05:45  oaumage
- quelques corrections au niveau du titre
- ajout de l'arborescence et de la disponibilite

Revision 1.2  2000/01/14 13:25:34  oaumage
- quelques corrections pour que le document compile

Revision 1.1  2000/01/14 13:10:54  oaumage
- Document Latex `Getting started with PM2'
