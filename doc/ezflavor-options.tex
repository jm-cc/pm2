\documentclass[11pt, a4paper ,twoside]{article}
\usepackage{epstopdf}
\usepackage{natbib}
\usepackage{epstopdf}
\usepackage[utf8]{inputenc}
\usepackage{a4wide}
\usepackage{rotating}
\usepackage{multicol}
\usepackage{afterpage}
\usepackage{color}
\usepackage{array}
\usepackage{floatflt}
\usepackage[colorlinks=true, linkcolor=blue, bookmarks=true, citecolor=blue, pagebackref]{hyperref}

\setcounter{secnumdepth}{3}
\setcounter{tocdepth}{1}
\usepackage[french]{minitoc}
\setcounter{parttocdepth}{2}
\usepackage{amsmath}

\usepackage[unitcntnoreset]{bibtopic}
\usepackage[final]{pdfpages}
\usepackage{amssymb}
\usepackage[english, french]{babel}
\pagestyle{plain}

\begin{document}
Il existe 2 types d'options pour chaque module:
\begin{itemize}
\item inclusives, cases carrées.
\item exclusives, cases rondes, choix unique par section.
\end{itemize}
~\\
 Pour chacune de ces options, il y a une option avec une boîte de texte permettant de fixer une valeur à cette option et même de définir une valeur par défaut.
La seule condition est de mettre un ``:'' à la fin du nom de cette option.
\\
On aura par exemple: \verb|--marcel-stacksize:64| (64 étant la valeur définie par défaut).
\\
Il faut alors rajouter les fichiers suivants dans \verb|nom-de-la-flavor/config/options|:
\begin{enumerate}
\item si inclusive
\begin{itemize}
\item \verb|0xx-nom-de-l'option:.sh|
\item \verb|0xx-nom-de-l'option:.dft| (pour définir une valeur par défaut)
\item \verb|0xx-nom-de-l'option:.dep| (si besoin de dépendances)
\item \verb|0xx-nom-de-l'option:.help| (indique ce que fait l'option)
\end{itemize}
\item si exclusive
\begin{itemize}
\item \verb|1xx-nom-de-l'option:.sh|
\item \verb|1xx-nom-de-l'option:.dft| (pour définir une valeur par défaut)
\item \verb|1xx-nom-de-l'option:.dep| (si besoin de dépendances)
\item \verb|1xx-nom-de-l'option:.help| (indique ce que fait l'option)
\end{itemize}
\end{enumerate}
~\\
Une fois cela fait, il suffit de taper la commande : \verb|pm2-recreate-links|.
~\\
Liste des commandes dans pm2 les plus courantes :
\begin{itemize}
\item \verb|pm2-flavor get --flavor=nom-de-la-flavor| : Affiche le contenu de la flavor
\item \verb|pm2-flavor list| : Liste les différentes flavor
\item \verb|pm2-flavor set --flavor=nom-de-la-flavor| : Si on ne met aucune option a coté du set, on vide toutes les options de la flavor et initialise tout à nulle.
\item \verb|pm2-flavor set --flavor=nom-de-la-flavor `cat FICHIER`| : Met le contenu du fichier dans la flavor
\item \verb|xargs pm2-flavor set --flavor=nom-de-la-flavor < FICHIER| : Met le contenu du fichier dans la flavor avec la classe
\item \verb|pm2-create-sample-flavor nom-de-la-flavor| : Régénère la flavor avec ses valeurs par défaut
\end{itemize}
\end{document}
