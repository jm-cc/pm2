\documentclass[11pt, a4paper ,twoside]{book} 
\usepackage{epstopdf}   
\usepackage{natbib}	
\usepackage{epstopdf}   
\usepackage[applemac]{inputenc}	
\usepackage{a4wide}
\usepackage{fancyheadings}	
\usepackage{rotating}
\usepackage{multicol}	
\usepackage{afterpage} 
\usepackage{color}
\usepackage{array}	
\usepackage{floatflt}	 
\usepackage[colorlinks=true, linkcolor=blue, bookmarks=true, citecolor=blue, pagebackref]{hyperref} 

\setcounter{secnumdepth}{3} 
\setcounter{tocdepth}{1} 
\usepackage[french]{minitoc} 
\setcounter{parttocdepth}{2} 
\usepackage{amsmath}

\usepackage[unitcntnoreset]{bibtopic}  
\usepackage[final]{pdfpages} 
\usepackage{amssymb}
\usepackage[english, french]{babel} 
\pagestyle{fancy}

\begin{document}
Il existe 2 types d'options pour chaque module:
\begin{itemize}
\item inclusives, cases carr�es.
\item exclusives, cases rondes, choix unique par section.
\end{itemize}
~\\
 Pour chacune de ces options, il y a une option avec une bo�te de texte permettant de fixer une valeur � cette option et m�me de d�finir une valeur par d�faut.
La seule condition est de mettre un ``:'' � la fin du nom de cette option.
\\
On aura par exemple: \texttt{--marcel-stacksize:64} (64 �tant la valeur d�finie par d�faut).
\\
Il faut alors rajouter les fichiers suivants dans \texttt{nom-de-la-flavor/config/options}:
\begin{enumerate}
\item si inclusive 
\begin{itemize}
\item \texttt{0xx-nom-de-l'option:.sh}
\item \texttt{0xx-nom-de-l'option:.dft} (pour d�finir une valeur par d�faut)
\item \texttt{0xx-nom-de-l'option:.dep} (si besoin de d�pendances)
\item \texttt{0xx-nom-de-l'option:.help} (indique ce que fait l'option)
\end{itemize}
\item si exclusive
\begin{itemize}
\item \texttt{1xx-nom-de-l'option:.sh}
\item \texttt{1xx-nom-de-l'option:.dft} (pour d�finir une valeur par d�faut)
\item \texttt{1xx-nom-de-l'option:.dep} (si besoin de d�pendances)
\item \texttt{1xx-nom-de-l'option:.help} (indique ce que fait l'option)
\end{itemize}
\end{enumerate}
~\\
Une fois cela fait, il suffit de taper la commande : \texttt{pm2-recreate-links}.
~\\
Liste des commandes dans pm2 les plus courantes :
\begin{itemize}
\item \texttt{pm2-flavor get --flavor=nom-de-la-flavor} : Affiche le contenu de la flavor
\item \texttt{pm2-flavor list} : Liste les diff�rentes flavor
\item \texttt{pm2-flavor set --flavor=nom-de-la-flavor} : Si on ne met aucune option a cot� du set, on vide toutes les options de la flavor et initialise tout � nulle.
\item \texttt{pm2-flavor set --flavor=nom-de-la-flavor `cat FICHIER`} : Met le contenu du fichier dans la flavor 
\item \texttt{xargs pm2-flavor set --flavor=nom-de-la-flavor < FICHIER} : Met le contenu du fichier dans la flavor avec la classe
\item \texttt{pm2-create-sample-flavor nom-de-la-flavor} : R�g�n�re la flavor avec ses valeurs par d�faut
\end{itemize}
\end{document}