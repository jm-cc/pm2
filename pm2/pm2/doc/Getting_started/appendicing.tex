\chapter{Additional PM2 material}

\stamp $Id: appendicing.tex,v 1.4 2001/04/18 15:13:12 bouge Exp $

\section{Using a standard main}
\label{sec:tradimain}

Some application may require to use the traditional \|main| function
name instead of the regular \|pm2_main|. This can be enabled through
the \|standard_main| options of the Marcel flavors. This option
defines the \|STANDARD_MAIN| compilation flag, tobe used for each of
the PM2 programs. Note that this may affect the performance of several
thread management functions. The user is therefore discouraged to do
this unless explicitly required.

\section{Synopsys of PM2 scripts}
\label{sec:scripts}

\subsection{\|pm2conf|}

\begin{shell}
ravel% pm2conf -h
Usage: pm2conf { option } { item }

option:
-f <name>: Use flavor "name" (default=$PM2_FLAVOR or default)
-p <name>: Store parameters in file "name" 
                              under the preference directory
-h: Display this help message

item:
<host>: Add machine "host" to the configuration
-l <file>: Use <file> as an input for host names
-e <host> <i>-<j>: Expand to <hosti> <hosti+1> ... <hostj>
-s <suffix> item: Add <suffix> to the "item" expression
\end{shell}

A typical use of the extended options could thus be:
\begin{shell}
ravel% pm2conf -s tcp cluster0 -s tcp -e cluster 10-12
0 : cluster0tcp
1 : cluster10tcp
2 : cluster11tcp
3 : cluster12tcp
\end{shell}

\subsection{\|pm2load|}

\begin{shell}
ravel% pm2load -h
Usage: pm2load { option } command

option:
-f name: Use flavor named "name" (default=$PM2_FLAVOR or default)
-d: Run command in debug mode 
                           (not supported by all implementations)
-c name: Use console named "name"
-l: Also generate a log file on first node
-L: Only generate a log file on first node
-h: Display this help message
\end{shell}

\begin{note}
  LB to RN: The -L option does not seem to work properly
\end{note}


\section{Common Pitfalls}
\label{sec:commonpitfalls}

Table~\ref{tbl:pitfalls} sums up the symptoms together with
possible solutions.
\begin{table}[p]
\caption{Common Pitfalls\label{tbl:pitfalls}}
\begin{center}
\begin{tabular}{|p{0.30\linewidth}|p{0.65\linewidth}|}                          \hline
Symptoms & 
Solutions                   
\\ 
\hline
\|Permission denied| error at load time & 
Each configuration node must be made
accessible by \|rsh| from the local host: update your
\|.rhosts| file                                  
\\ 
\hline
\end{tabular}
\end{center}
\end{table}

\section{Frequently asked questions}
\label{sec:faq}


\begin{quote}
  Why do I get the message "Permission denied" when running pm2load?
  
  Why does my program crash the entire cluster?
\end{quote}
