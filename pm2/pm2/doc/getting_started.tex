\documentclass[a4paper,11pt]{report}

% packages
\usepackage{moreverb}
% \usepackage{fullpage}
\usepackage{boxedminipage}
\usepackage[dvips]{graphics}
\usepackage{url}
\usepackage{psboxit}
\usepackage{xspace}
\usepackage{oldgerm}
\usepackage{afterpage}
% \usepackage{color}

% settings
\let\bfseriesaux=\bfseries \def\bfseries{\sffamily\bfseriesaux}
% \definecolor{jaune}{rgb}{1,1,.85}
% \definecolor{bleu}{rgb}{0, 0, .4}
% \definecolor{bleu2}{rgb}{0, 0, .6}
% \definecolor{pourpre}{rgb}{0.4, 0, 0.4}
% \definecolor{orange1}{rgb}{1, .5, 0}
% \definecolor{orange2}{rgb}{1, .7, 0}
% \pagecolor{jaune}
% \color{bleu2}

% macros
\def\pm2{PM$^2$\xspace}
\def\mad{\emph{Madeleine}\xspace}
\def\madii{\emph{Madeleine~II}\xspace}
\def\mar{\emph{Marcel}\xspace}
\newcommand{\txty}{\fontfamily{yinit}\fontencoding{U}\fontseries{m}%
\fontshape{n}\selectfont}
\newcommand{\txtcmdh}{\fontfamily{cmdh}\fontencoding{T1}\fontseries{m}%
\fontshape{n}\selectfont}
\newcommand{\txtcmfr}{\fontfamily{cmfr}\fontencoding{T1}\fontseries{m}%
\fontshape{n}\selectfont}
% \def\aut{\txtcmfr}
\def\aut{\frakfamily}
\def\mp{\scriptsize\raggedright}
\def\endchap{\vspace{2cm}\begin{center}\rule{6cm}{1mm}\end{center}}

% first page
%\title{\color{pourpre}{\fontfamily{yinit}\fontencoding{U}\fontseries{m}%
%\fontshape{n}\selectfont G}\gothfamily\Huge etting Started with PM2}
%\author{\color{bleu} Gabriel Antoniu \and%
%        \color{bleu} Olivier Aumage \and%
%        \color{bleu} Luc Boug\'e \and%
%        \color{bleu} Jean-Fran\c{c}ois M\'ehaut \and%
%        \color{bleu} Raymond Namyst}
\title{{\txty G}\gothfamily\Huge etting Started with PM2}
\author{\aut {\txty G}abriel Antoniu \and%
        \aut {\txty O}livier Aumage \and%
        \aut {\txty L}uc Boug\'e \and%
        \aut {\txty J}ean-Fran\c{c}ois M\'ehaut \and%
        \aut {\txty R}aymond Namyst}
\date{\swabfamily\small{\txty F}eb. 2000}

%%%%%%%%%%%%%%%%
\begin{document}
\makeatletter
\begin{titlepage}
  \let\footnotesize\small
  \let\footnoterule\relax
  \let \footnote \thanks
  \null\vfil
  \vskip 60\p@
  \begin{center}%
    {\setlength{\unitlength}{5mm}%
    \begin{picture}(24, 1)(0, 0)%
      \linethickness{2pt}%
      \qbezier(24.01,0.99)(22.00,-16.00)(12.00,-24.00)%
      \qbezier(21.99,1.00)(24.00,1.50)(24.80,2.00)%
      \put(7.00,0.99){\line(1,0){15}}
      \linethickness{0.4pt}%
      \qbezier(24.00,1.00)(12.00,-3.00)(0.00,1.00)%
      \qbezier(24.00,1.00)(22.00,-16.00)(12.00,-24.00)%
      \qbezier(0.00,1.00)(2.00,-16.00)(12.00,-24.00)%
      \put(12.00,-1.00){\line(0,-1){23}}
      \put(7.00,3.00){\line(1,0){15}}
      \put(7.00,2.00){\line(1,0){18}}
      \put(7.00,1.00){\line(1,0){15}}
      \qbezier(22.00,3.00)(24.00,2.50)(25.00,2.00)%
      \qbezier(22.00,1.00)(24.00,1.50)(25.00,2.00)%
    \end{picture}%
    \vskip -22mm
    \LARGE \@title \par}%
    \vskip 3em%
    {\large
     \lineskip .75em%
      \begin{tabular}[t]{c}%
        \@author
      \end{tabular}\par}%
      \vskip 5em%<
    {\large \@date \par}%       % Set date in \large size.
  \end{center}\par
  \@thanks
  \vfil\null
\end{titlepage}
\makeatother
%\maketitle
{\txtcmfr
\tableofcontents
}
%%%%%%%%%%%%%%%%%%%%%%%%%%%%%%%%%%%%%%%%%%%%%%%%%%%%%%%%%%%%%%%%%%%%%%%%%%%%%%
\chapter{Introducing \pm2}

% Introduction
%______________
\section{Introduction : What is \pm2?}
The \pm2 environment is a joint project between the \textsc{LIP}
(\emph{\'Ecole Normale Sup\'erieure de Lyon}, France) and the \textsc{LIFL}
(\emph{Universit\'e des Sciences et Technologies de Lille}, France). 

%===========
\subsection{Multithreaded Programming Environment}
\marginpar{\mp API (C language) + set of libraries}
\pm2 (Parallel Multithreaded Machine) is a distributed multithreaded
environment designed to efficiently support irregular parallel
applications on distributed architectures. \pm2 can manage several
hundreds of threads on each available physical processor. The \pm2
interface provides functionalities for the management of this high
degree of parallelism and for dynamic load balancing. Distinguishing
features of \pm2 include its priority driven scheduling policy, its
thread migration mechanisms and its ability to ease the development
of various load balancing policies.

%===========
\subsection{Programming model}
The \pm2 distributed multithreaded programming environment easily
helps the developer to express the parallelism out of his/her programs
by the mean of the \emph{remote procedure call} paradigm (RPC). Each
node of a cluster running \pm2 can be considered as being both a
client and a service provider. While remote requests can be serviced
on the fly by the server thread of each process, additionnal threads
may be spawned on demand to handle more computation intensive remote
client service requests.

% Installation
%_____________________________________________________________________________
\section{Installation}

%===========
\subsection{Currently Supported Platforms}
\pm2 is a highly portable and efficient environment and the current
software is yet available on a wide range of architectures. The
implementation is built on top of two separate software components:
\mar and \madii. 
\subsubsection{Multithreading}
\mar is a POSIX-compliant thread package that
provides extra-features such as thread migration that are needed by
the \pm2 runtime. \mar is currently available on \textsc{Pentium},
\textsc{MIPS}, \textsc{Alpha}, \textsc{Sparc} and \textsc{PowerPC}
processors. The Table~\ref{tbl:platforms} sums up the currently
supported platforms. 
\subsubsection{Communications}
\madii is a generic communication interface which is able to fully
exploit the low latency and the high bandwidth of high-speed networks
such as Myrinet or SCI. This \pm2 network subsystem currently supports
the protocols TCP, MPI (LAM-MPI, MPI-BIP), VIA, SISCI and SBP.
\begin{table}
\caption{Supported Platforms\label{tbl:platforms}}
\begin{center}
\begin{tabular}[p]{|l|l|}                                        \hline
Operating System & Architecture                               \\ \hline
\textsc{Linux}   & \textsc{x86}, \textsc{PPC}, \textsc{Alpha} \\
\textsc{Solaris} & \textsc{Sparc}, \textsc{x86}               \\
\textsc{FreeBSD} & \textsc{x86}                               \\
\textsc{Aix}     & \textsc{RS6000}                            \\
\textsc{Irix}    & \textsc{MIPS}                              \\
\textsc{OSF}     & \textsc{Alpha}                             \\
\textsc{Unicos}  & \textsc{Alpha}                             \\ \hline
\end{tabular}
\end{center}
\end{table}
\subsubsection{Required development tools}
As previously mentionned, \pm2 as been designed to be easily portable
and only relies on the availability of the two following development tools:
\begin{itemize}
\small
\item GNU C Compiler \textsc{gcc} (versions 2.7.2.3, 2.8.1 and \textsc{Egcs}
versions have been successfully tested).
\item GNU Make (versions 3.7x.x or more).
\end{itemize}
 

%===========
\subsection{Installing \pm2}

\subsubsection{Getting the \pm2 software and help}
The \pm2 archive is available from the \pm2 web site at:%
\begin{quote}
\url{http://www.ens-lyon.fr/~rnamyst/pm2.html} 

(contact guru:~\url{bouge@ens-lyon.fr}).
\end{quote}

\subsubsection{Unpacking the \pm2 distribution}
\marginpar{\mp $\rightarrow$ making a personal installation}
The \pm2 archive can be extracted with one of the following commands
entered at the shell prompt:
\begin{small}
\begin{verbatim}
        > tar zxvf pm2.tar.gz                # with the GNU tar

        > gunzip -c pm2.tar.gz | tar xvf -   # with otherwise
\end{verbatim}
\end{small}
Once extracted, the \pm2 distribution should be available under the
\texttt{./pm2/} directory. This \pm2 distribution is organized as
shown on Figure~\ref{fig:pm2-tree}.
\begin{figure}[p]
\begin{center}
\begin{boxedminipage}{0.9\textwidth}
\footnotesize
\begin{verbatim}

  pm2 --+
        |
        +-- bin........: PM2 scripts
        |  
        +-- doc........: documentation files
        |
        +-- dsm........: DSM-PM2 distributed shared memory manager
        |
        +-- examples...: variety of PM2 examples
        |
        +-- include....: PM2 header files
        |
        +-- lib........: PM2 library destination
        |
        +-- mad2.......: network subsystem
        |
        +-- make.......: PM2 makefiles
        |
        +-- marcel.....: multithreading management subsystem
        |
        +-- source.....: PM2 source files
        |
        +-- toolbox....: general purpose library              

\end{verbatim}
\end{boxedminipage}
\end{center}
\caption{\pm2 distribution tree structure\label{fig:pm2-tree}}
\end{figure}

\subsubsection{Environment variables}
\marginpar{\mp PM2\_ROOT, PATH, ...}  Several environment
variables should be set properly for \pm2 to work correctly. The
variable \texttt{PM2\_ROOT} should contain the path to the \pm2
distribution root directory. The variables \texttt{MARCEL\_ROOT} and
\texttt{MAD2\_ROOT} should also be set to the root directory of \mar
and \madii respectively. The following should be appended to your
\texttt{PATH} environment variable: 
\begin{quote}
\texttt{\$\{PM2\_ROOT\}/bin; \$\{MARCEL\_ROOT\}/bin; \$\{MAD2\_ROOT\}/bin;}.
\end{quote}

%===========
\subsection{Configuring\label{subsec:configuring}}
\marginpar{\mp no need to specify OS/processor}
Configuring \pm2 is quite straightforward. The underlying platform is
automatically detected. Hence, there is no need to specify the
operating sytem/processor pair.

\marginpar{\mp\texttt{pm2custom}} The only major setting
left to the user is the network protocol selection, which is
currently not desirable to be auto-detected by the \madii
communication library of \pm2. The network protocol choice is made
using the \texttt{pm2custom} command at the shell prompt. Example:
\begin{small}
\begin{verbatim}
        > pm2custom tcp
        The current network interface is set to: tcp
\end{verbatim}
\end{small}
Each time the network protocol setting is changed using
\texttt{pm2custom}, the \pm2 environment should be recompiled
(see~\ref{subsec:compiling}, \emph{Compiling}).

\subsubsection{Advanced configuration}
Advanced configuration options are of course available for the more
demanding users. These options should be used with great care. They can be
found at the top of the main \pm2 makefile, in the file
\texttt{\$\{PM2\_ROOT\}/make/common.mak}.

%===========
\subsection{Compiling\label{subsec:compiling}}
\marginpar{\mp not really required, but can be done by typing "cd pm2;
make"} The compilation step is not really required and rather should
be integrated into the targeted application build operation. This to
ensure that the application always gets linked with the up-to-date
versions of the \pm2 libraries (especially when changing the network
protocol configuration with \texttt{pm2custom},
see~\ref{subsec:configuring}, \emph{configuring}). Yet, it can be done
as a standalone step by typing \texttt{make} at the shell prompt into
the root directory of the \pm2 distribution.

\subsubsection{Cleanning the directory tree}
\marginpar{\mp pm2clean} The directory tree of the \pm2 distribution
may be cleaned at any time from compiled objects and libraries using
the command \texttt{pm2clean}. This may be needed to free some disk
space when \pm2 is not in use or to force the whole \pm2 distribution
to be completely remade by a subsequent \texttt{make} call.
\endchap

%%%%%%%%%%%%%%%%%%%%%%%%%%%%%%%%%%%%%%%%%%%%%%%%%%%%%%%%%%%%%%%%%%%%%%%%%%%%%%
\chapter{Discovering \pm2}

% PM2 Step by Step
%_____________________________________________________________________________
\section{\pm2 step by step}

%===========
\subsection{The minimal PM2 program}
\begin{figure}[p]
\begin{center}
\begin{boxedminipage}{0.8\textwidth}
\begin{footnotesize}
\begin{listing}{0}
 #include <stdio.h>
 #include <pm2.h>

 int pm2_main(int argc, char **argv)
 {
   pm2_init(&argc, argv);

   printf("Hello World !\n");
  
   if (pm2_self() == 0)
     pm2_halt();
  
   pm2_exit();
   return 0;
 }
\end{listing}
\end{footnotesize}
\end{boxedminipage}
\end{center}
\caption{Minimal \pm2 program\label{fig:ex1}}
\end{figure}
\afterpage{\clearpage}
Let us start our \pm2 tour by writting the minimal \pm2 program. The
Figure~\ref{fig:ex1} (on page~\pageref{fig:ex1}) shows an example of
such a minimalist \pm2 code. As we can see, a \pm2 program must first
include the \texttt{pm2.h} header file along with other standard
header files.

\subsubsection{Program body}
We now write the program body. The Line~7 illustrates that the start
function name should be \texttt{pm2\_main} instead of the classical
\texttt{main} when using \pm2. Yet, the arguments remain the regular
\texttt{argc}/\texttt{argv} pair. A call to \texttt{pm2\_init}
effectively initializes the \pm2 runtime system. Once the
\texttt{pm2\_init} function returns, the actual application specific
\emph{parallel processing} can begin. Here, this processing reduces to
a call to the \texttt{printf} function of the standard library. This
call will be concurrently performed on any process constituting the
session configuration during the program execution.

\subsubsection{Session termination}
Finally, the termination phasis concludes our minimal \pm2 program. It
is splitted into two steps. One (and only one) of the nodes must first
perform a call to \texttt{pm2\_halt}. This step ensures that no
further incoming service request will be accepted by any of the
session processes. The \texttt{pm2\_self} function is used to retrieve
the local process rank. Hence, in the example the \texttt{pm2\_halt}
call is executed by the process \texttt{0} (sometimes called the
\emph{master} process). Once each process reaches the
\texttt{pm2\_exit} second termination step, the \pm2 session ends.

%===========
\subsection{Running a \pm2 program}
\subsubsection{Session topology definition}
We now learn how to run our \pm2 program. The first required step is
to specify the list of hostnames on which the \pm2 session is going to
span. For this we use the \texttt{pm2conf} command. Example for a
session using 3 nodes on a cluster called \texttt{POPC}:
\begin{verbatim}
    > pm2conf popc1 popc2 popc3
    The current MADELEINE configuration contains 4 host(s) :
    0 : popc1
    1 : popc2
    2 : popc3
\end{verbatim}
The node \texttt{popc1} is the master of the session: it owns the
rank~0. The other nodes are the slaves. 

\subsubsection{Launching the program}
Although not a requirement in general, some protocols may expect the
use of proprietary startup scripts. They may also expect other
prerequities like the session to be launched from the master node. The
\pm2 script \texttt{pm2load} takes care about all the details of
setting up the \pm2 session, whatever the underlying network protocol
used. This script takes the application program name as the only
parameter.

If the compiled executable of our sample \pm2 program is
\texttt{example1}, then it can be launched by typing the following
command at the prompt:
\begin{verbatim}
    > pm2load ./example1
    Hello World !
    [Threads : 4 created, 0 imported (0 cached)]
\end{verbatim}
We can see that our program generates two messages. The first one is the
expected \emph{Hello world !} message coming from the master node
\texttt{printf}. The second one is the multithreads manager stats for
the session.

\subsubsection{Results from other nodes}
One can remark that the previous example only printed the message
generated by the master node process. Indeed, the standard output of
the other processes is redirected to log files located into the /tmp
directory of each node. The logs are easily accessible using the
command \texttt{pm2logs} which is charged to retrieve and display logs
from each slave node of the session configuration.

%===========
\subsection{Common pitfalls}


%===========
\subsection{Getting the output in the console}
\begin{figure}[p]
\begin{center}
\begin{boxedminipage}{0.8\textwidth}
\begin{footnotesize}
\begin{listing}{0}
 #include <stdio.h>
 #include <pm2.h>

 int pm2_main(int argc, char **argv)
 {
   pm2_init(&argc, argv);

   if (pm2_self() == 1)
     {
       printf("Hello World !\n");    
       pm2_halt();
     }
  
   pm2_exit();
   return 0;
 }
\end{listing}
\end{footnotesize}
\end{boxedminipage}
\end{center}
\caption{Output redirection, example 1\label{fig:ex2-1}}
\end{figure}

\begin{figure}[p]
\begin{center}
\begin{boxedminipage}{0.8\textwidth}
\begin{footnotesize}
\begin{listing}{0}
 #include <stdio.h>
 #include <pm2.h>

 int pm2_main(int argc, char **argv)
 {
   pm2_init(&argc, argv);

   if (pm2_self() == 1)
     {
       pm2_printf("Hello World !\n");    
       pm2_halt();
     }
  
   pm2_exit();
   return 0;
 }
\end{listing}
\end{footnotesize}
\end{boxedminipage}
\end{center}
\caption{Output redirection, example 2\label{fig:ex2-2}}
\end{figure}

\afterpage{\clearpage}

%===========
\subsection{Invoking a remote service}

\begin{figure}[p]
\begin{center}
\begin{boxedminipage}{0.8\textwidth}
\begin{footnotesize}
\begin{listing}{0}
 #include <stdio.h>
 #include <pm2.h>

 static unsigned int service_id;

 static void service(void)
 {
   pm2_rawrpc_waitdata();
   printf("Hello, World!\n");
 }

 int pm2_main(int argc, char **argv)
 {
   pm2_rawrpc_register(&service_id, service)
   pm2_init(&argc, argv);

   if (pm2_self() == 0)
     {
       pm2_rawrpc_begin(1, service_id, NULL);
       pm2_rawrpc_end();

       pm2_halt();
     }
   pm2_exit();
   return 0;
 }
\end{listing}
\end{footnotesize}
\end{boxedminipage}
\end{center}
\caption{Sample service\label{fig:ex3}}
\end{figure}

\afterpage{\clearpage}

%===========
\subsection{Passing parameters}

\begin{figure}[p]
\begin{center}
\begin{boxedminipage}{0.9\textwidth}
\begin{footnotesize}
\begin{listing}{0}
#include <stdio.h>
#include <pm2.h>

static unsigned int service_id;

static void service(void)
{
  int len;
  char *s;

  pm2_unpack_int(SEND_CHEAPER, RECV_EXPRESS, &len, 1);
  s = malloc(len);
  pm2_unpack_byte(SEND_CHEAPER, RECV_CHEAPER, s, len);
  pm2_rawrpc_waitdata();

  printf("The sentence is %s\n", s);
}

int pm2_main(int argc, char **argv)
{
  pm2_rawrpc_register(&service_id, service)
  pm2_init(&argc, argv);
  if (pm2_self() == 0)
    {
      char s[] = "A la recherche du temps perdu.";
      int len;

      len = strlen(s) + 1;
      pm2_rawrpc_begin(1, service_id, NULL);
      pm2_pack_int(SEND_CHEAPER, RECV_EXPRESS, &len, 1);
      pm2_pack_byte(SEND_CHEAPER, RECV_CHEAPER, s, len);
      pm2_rawrpc_end();

      pm2_halt();
    }
  pm2_exit();
  return 0;
}
\end{listing}
\end{footnotesize}
\end{boxedminipage}
\end{center}
\caption{Service with a string parameter\label{fig:ex4}}
\end{figure}

\afterpage{\clearpage}

\subsubsection{Packing/unpacking API}
        pack\_int/pack\_byte
        SEND\_CHEAPER, etc.

\subsubsection{A more interesting example}

%===========
\subsection{"Threaded" services}
        Services handlers are executed sequentially
        Thread creation: pm2\_thread\_create
        Caveat! pm2\_waitdata...
\endchap

%%%%%%%%%%%%%%%%%%%%%%%%%%%%%%%%%%%%%%%%%%%%%%%%%%%%%%%%%%%%%%%%%%%%%%%%%%%%%%
\chapter{Mastering \pm2}

% Advandced Threading
%_____________________________________________________________________________
\section{Advanced thread features}

%===========
\subsection{Thread synchronization}

%===========
\subsection{Invoking services synchronously}
        2 one-way invocations + synchronization

\subsubsection{Examples}
        Better: using the "pm2\_rawrpc\_wait" features

%===========
\subsection{Thread migration}
        Thread state : migratable/non-...

\subsubsection{Migration in 3 steps}
        freezing the local scheduler
        getting the list of migratable threads
        selecting and migrating threads

\subsubsection{Example}
        Iso-address migration

%===========
\subsection{Migrating with dynamically allocated data}

\subsubsection{Isomalloc}


% DSM
%_____________________________________________________________________________
\section{Sharing data across clusters}
        DSM-PM2
\endchap

%%%%%%%%%%%%%%%%%%%%%%%%%%%%%%%%%%%%%%%%%%%%%%%%%%%%%%%%%%%%%%%%%%%%%%%%%%%%%%
\chapter{Additional \pm2 material}

% FAQ
%_____________________________________________________________________________
\section{FAQ}
        Why does my program crash the entire cluster?
        Why do I get the message "Permission denied" when running pm2load?


% Bibliography
%_____________________________________________________________________________
\section{Bibliography}
    
\endchap
            
\end{document}
_______________________________________________________________________________
$Log: getting_started.tex,v $
Revision 1.13  2000/01/26 17:38:32  oaumage
- quelques modifs ...

Revision 1.12  2000/01/26 11:30:53  oaumage
- exemple 1 complete
- diverses corrections et modifications

Revision 1.11  2000/01/25 17:11:31  oaumage
- diverses modifications et corrections

Revision 1.10  2000/01/25 10:15:50  oaumage
- passage en mode Report: il semble que se soit plus facile a lire, et il
  s'agit du style generalement employe pour une documentation

Revision 1.9  2000/01/25 10:03:06  oaumage
- remplissage de la premiere partie (intro/installation)
- un petit clin d'oeil pour la page de titre

Revision 1.8  2000/01/21 17:26:00  oaumage
- modification des exemples pour tenir compte de l'evolution des fonctions
  pack/unpack

Revision 1.7  2000/01/17 13:36:00  oaumage
- correction des exemples 2-1 et 2-2

Revision 1.6  2000/01/14 17:16:00  oaumage
- ajout des exemples 2, 3 et 4

Revision 1.5  2000/01/14 16:06:15  oaumage
- exemple minimal + texte

Revision 1.4  2000/01/14 15:38:05  oaumage
- nouveau programme minimal

Revision 1.3  2000/01/14 14:05:45  oaumage
- quelques corrections au niveau du titre
- ajout de l'arborescence et de la disponibilite

Revision 1.2  2000/01/14 13:25:34  oaumage
- quelques corrections pour que le document compile

Revision 1.1  2000/01/14 13:10:54  oaumage
- Document Latex `Getting started with PM2'
