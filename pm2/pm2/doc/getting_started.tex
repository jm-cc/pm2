\documentclass[a4paper,11pt]{report}

% packages
\usepackage{moreverb}
% \usepackage{fullpage}
\usepackage{boxedminipage}
\usepackage[dvips]{graphics}
\usepackage{url}
\usepackage{psboxit}
\usepackage{xspace}
\usepackage{oldgerm}
\usepackage{afterpage}
% \usepackage{color}

% settings
\let\bfseriesaux=\bfseries \def\bfseries{\sffamily\bfseriesaux}
% \definecolor{jaune}{rgb}{1,1,.85}
% \definecolor{bleu}{rgb}{0, 0, .4}
% \definecolor{bleu2}{rgb}{0, 0, .6}
% \definecolor{pourpre}{rgb}{0.4, 0, 0.4}
% \definecolor{orange1}{rgb}{1, .5, 0}
% \definecolor{orange2}{rgb}{1, .7, 0}
% \pagecolor{jaune}
% \color{bleu2}

% macros
\def\pmm{PM$^2$\xspace}
\def\mad{\emph{Madeleine}\xspace}
\def\madii{\emph{Madeleine~II}\xspace}
\def\mar{\emph{Marcel}\xspace}
\newcommand{\txty}{\fontfamily{yinit}\fontencoding{U}\fontseries{m}%
\fontshape{n}\selectfont}
\newcommand{\txtcmdh}{\fontfamily{cmdh}\fontencoding{T1}\fontseries{m}%
\fontshape{n}\selectfont}
\newcommand{\txtcmfr}{\fontfamily{cmfr}\fontencoding{T1}\fontseries{m}%
\fontshape{n}\selectfont}
% \def\aut{\txtcmfr}
\def\aut{\frakfamily}
\def\mp{\scriptsize\raggedright}
\def\endchap{\vspace{2cm}\begin{center}\rule{6cm}{1mm}\end{center}}

% first page
%\title{\color{pourpre}{\fontfamily{yinit}\fontencoding{U}\fontseries{m}%
%\fontshape{n}\selectfont G}\gothfamily\Huge etting Started with PM2}
%\author{\color{bleu} Gabriel Antoniu \and%
%        \color{bleu} Olivier Aumage \and%
%        \color{bleu} Luc Boug\'e \and%
%        \color{bleu} Jean-Fran\c{c}ois M\'ehaut \and%
%        \color{bleu} Raymond Namyst}
\title{{\txty G}\gothfamily\Huge etting Crazy with PM2}
\author{\aut {\txty G}abriel Antoniu \and%
        \aut {\txty O}livier Aumage \and%
        \aut {\txty L}uc Boug\'e \and%
        \aut {\txty J}ean-Fran\c{c}ois M\'ehaut \and%
        \aut {\txty R}aymond Namyst}
\date{\swabfamily\small{\txty F}eb. 2000}

%%%%%%%%%%%%%%%%
\begin{document}
\makeatletter
\begin{titlepage}
  \let\footnotesize\small
  \let\footnoterule\relax
  \let \footnote \thanks
  \null\vfil
  \vskip 60\p@
  \begin{center}%
    {\setlength{\unitlength}{5mm}%
    \begin{picture}(24, 1)(0, 0)%
      \linethickness{2pt}%
      \qbezier(24.01,0.99)(22.00,-16.00)(12.00,-24.00)%
      \qbezier(21.99,1.00)(24.00,1.50)(24.80,2.00)%
      \put(7.00,0.99){\line(1,0){15}}
      \linethickness{0.4pt}%
      \qbezier(24.00,1.00)(12.00,-3.00)(0.00,1.00)%
      \qbezier(24.00,1.00)(22.00,-16.00)(12.00,-24.00)%
      \qbezier(0.00,1.00)(2.00,-16.00)(12.00,-24.00)%
      \put(12.00,-1.00){\line(0,-1){23}}
      \put(7.00,3.00){\line(1,0){15}}
      \put(7.00,2.00){\line(1,0){18}}
      \put(7.00,1.00){\line(1,0){15}}
      \qbezier(22.00,3.00)(24.00,2.50)(25.00,2.00)%
      \qbezier(22.00,1.00)(24.00,1.50)(25.00,2.00)%
    \end{picture}%
    \vskip -22mm
    \LARGE \@title \par}%
    \vskip 3em%
    {\large
     \lineskip .75em%
      \begin{tabular}[t]{c}%
        \@author
      \end{tabular}\par}%
      \vskip 5em%<
    {\large \@date \par}%       % Set date in \large size.
  \end{center}\par
  \@thanks
  \vfil\null
\end{titlepage}
\makeatother
%\maketitle
{\txtcmfr
\tableofcontents
}
%%%%%%%%%%%%%%%%%%%%%%%%%%%%%%%%%%%%%%%%%%%%%%%%%%%%%%%%%%%%%%%%%%%%%%%%%%%%%%
\chapter{Introducing \pmm}

% Introduction
%______________
\section{Introduction : What is \pmm?}

The \pmm project is conducted at the \emph{Laboratoire de
  l'Informatique du Parall�lisme} (\textsc{LIP}) which is part of the
(\emph{\'Ecole Normale Sup\'erieure de Lyon} (France). It aims at
providing an efficient programming environment for the support of high
demanding parallel applications on top of distributed architectures.
  
\pmm (Parallel Multithreaded Machine) is a distributed multithreaded
environment designed to efficiently support irregular parallel
applications on distributed architectures. \pmm can manage several
hundreds of threads on each available physical processor. The \pmm
interface provides functionalities for the management of this high
degree of parallelism and for dynamic load balancing. Distinguishing
features of \pmm include its priority driven scheduling policy, its
thread migration mechanisms and its ability to ease the development of
various load balancing policies.

\subsection{Programming model}

The \pmm distributed multithreaded programming environment easily
helps the developer to express the parallelism out of his/her programs
by the mean of the \emph{remote procedure call} paradigm (RPC). Each
node of a cluster running \pmm can be considered as being both a
client and a service provider. While remote requests can be serviced
on the fly by the server thread of each process, additionnal threads
may be spawned on demand to handle more computation intensive remote
client service requests.

% Installation
%_____________________________________________________________________________
\section{Installation}

Before getting into describing the installation process of \pmm, we
briefly review the currently supported architectures and systems. For
more up-to-date information, please refer to the
\verb|pm2/ARCHITECTURES| file within the software distribution.

\subsection{Currently Supported Platforms}

\pmm is a highly portable and efficient environment and the current
software is yet available on a wide range of architectures. The
implementation is built on top of two separate software components:
\mar and \madii. 

\subsubsection{Multithreading}

\mar is a POSIX-compliant thread package that
provides extra-features such as thread migration that are needed by
the \pmm runtime. \mar is currently available on \textsc{Pentium},
\textsc{MIPS}, \textsc{Alpha}, \textsc{Sparc} and \textsc{PowerPC}
processors. The Table~\ref{tbl:platforms} sums up the currently
supported platforms. 

\subsubsection{Communications}

\madii is a generic communication interface which is able to fully
exploit the low latency and the high bandwidth of high-speed networks
such as Myrinet or SCI. This \pmm network subsystem currently supports
the protocols TCP, MPI (LAM-MPI, MPI-BIP), VIA, SISCI and SBP.
\begin{table}
\caption{Supported Platforms\label{tbl:platforms}}
\begin{center}
\begin{tabular}[p]{|l|l|}                                        \hline
Operating System & Architecture                               \\ \hline
\textsc{Linux}   & \textsc{x86}, \textsc{PowerPC}, \textsc{Alpha} \\
\textsc{Solaris} & \textsc{Sparc}, \textsc{x86}               \\
\textsc{FreeBSD} & \textsc{x86}                               \\
\textsc{Aix}     & \textsc{RS6000}                            \\
\textsc{Irix}    & \textsc{MIPS}                              \\
\textsc{OSF}     & \textsc{Alpha}                             \\
\textsc{Unicos}  & \textsc{Alpha}                             \\ \hline
\end{tabular}
\end{center}
\end{table}

\subsubsection{Required development tools}

As previously mentionned, \pmm as been designed to be easily portable
and only relies on the availability of the two following development tools:
\begin{itemize}
\small
\item GNU C Compiler \textsc{gcc} (versions 2.7.2.3, 2.8.1 and \textsc{Egcs}
versions have been successfully tested).
\item GNU Make (versions 3.7x.x or more).
\end{itemize}
 
\subsection{Installing \pmm}

The following sections illustrate how to install \pmm on your machine.

\subsubsection{Getting the \pmm software and help}

The \pmm archive containing all the necessary sources is available
from the \pmm web site at:%
\begin{quote}
    \url{http://www.ens-lyon.fr/~rnamyst/pm2.html}
\end{quote}

You send an email to \url{bouge@ens-lyon.fr} if you encounter problems
with the PM2 software. However, please make sure that your problem is
not already listed in the \emph{Frequently Asked Questions} section
(Section \ref{sec:faq}, page~\pageref{sec:faq}).

\subsubsection{Unpacking the \pmm distribution}

\marginpar{\mp $\rightarrow$ making a personal installation}
The \pmm archive can be extracted with one of the following commands
entered at the shell prompt:
\begin{small}
\begin{verbatim}
        > tar xvfz pm2.tar.gz                # with the GNU tar

        > gunzip -c pm2.tar.gz | tar xvf -   # with otherwise
\end{verbatim}
\end{small}
Once extracted, the \pmm distribution should be available under the
\texttt{./pm2/} directory. This \pmm distribution is organized as
shown on Figure~\ref{fig:pm2-tree}.
\begin{figure}%[p]
\begin{center}
\begin{boxedminipage}{0.9\textwidth}
\footnotesize
\begin{verbatim}

  pm2 --+
        |
        +-- bin........: PM2 scripts
        |
        +-- doc........: documentation files
        |
        +-- dsm........: DSM-PM2 distributed shared memory manager
        |
        +-- examples...: variety of PM2 examples
        |
        +-- include....: PM2 header files
        |
        +-- lib........: PM2 library destination
        |
        +-- mad1.......: old network subsystem (for backward
        |                compatibility)
        +-- mad2.......: network subsystem
        |
        +-- make.......: PM2 makefiles
        |
        +-- marcel.....: multithreading management subsystem
        |
        +-- source.....: PM2 source files
        |
        +-- toolbox....: general purpose library

\end{verbatim}
\end{boxedminipage}
\end{center}
\caption{\pmm distribution tree structure\label{fig:pm2-tree}}
\end{figure}

\subsubsection{Environment variables}

Two environment variables have to be set so that \pmm will work
correctly. First, the \texttt{PM2\_ROOT} variable must contain the
path to the \pmm distribution root directory. Second, the
\texttt{\$\{PM2\_ROOT\}/bin} directory has to be included in the
current search path.

Please make sure that these two variables are set correctly within
remote shell commands (\emph{i.e.}, scripts invoked by \verb+rsh+).
Thus, you may insert the following lines in your \verb+.cshrc+ file
(if your default shell is \texttt{csh}):
\begin{small}
\begin{verbatim}
    setenv PM2_ROOT ${HOME}/pm2
    setenv PATH ${PATH}:${PM2_ROOT}/bin
\end{verbatim}%$
\end{small}

\subsection{Configuring\label{subsec:configuring}}

Configuring \pmm is quite straightforward. The underlying platform is
automatically detected. Hence, there is no need to specify the
operating sytem/processor pair.

The only major setting left to the user is the network protocol
selection, which is currently not desirable to be auto-detected by the
\madii communication library of \pmm. The network protocol choice is
made using the \texttt{pm2custom} command at the shell prompt.
Example:
\begin{small}
\begin{verbatim}
        > pm2custom tcp
        The current network interface is set to: tcp
\end{verbatim}
\end{small}
Each time the network protocol setting is changed using
\texttt{pm2custom}, the \pmm environment should be recompiled
(see~\ref{subsec:compiling}, \emph{Compiling}).

\subsubsection{Advanced configuration}

Advanced configuration options are of course available for the more
demanding users. These options should be used with great care. They
can be found at the top of the main \pmm makefile, in the file
\texttt{\$\{PM2\_ROOT\}/make/common.mak}.

\subsection{Compiling\label{subsec:compiling}}

The compilation step is not really required and rather should be
integrated into the targeted application build operation. This to
ensure that the application always gets linked with the up-to-date
versions of the \pmm libraries (especially when changing the network
protocol configuration with \texttt{pm2custom},
see~\ref{subsec:configuring}, \emph{configuring}). Yet, it can be done
as a standalone step by typing \texttt{make} at the shell prompt into
the root directory of the \pmm distribution.

\subsubsection{Cleanning the directory tree}

The directory tree of the \pmm distribution may be cleaned at any time
from compiled objects and libraries using the command
\texttt{pm2clean}. This may be needed to free some disk space when
\pmm is not in use or to force the whole \pmm distribution to be
completely remade by a subsequent \texttt{make} call.

Note that you can accelerate the cleaning process by specifying the
target architecture as an argument. For instance, the following
command will only clean object files, libraries and executable related
to the Intel Linux architecture :
\begin{small}
\begin{verbatim}
       > pm2clean LINUX/X86
\end{verbatim}
\end{small}

\endchap

%%%%%%%%%%%%%%%%%%%%%%%%%%%%%%%%%%%%%%%%%%%%%%%%%%%%%%%%%%%%%%%%%%%%%%%%%%%%%%
\chapter{Discovering \pmm}

This chapter aims at illustrating the basic features of \pmm through
some very simple programs. Next chapter will present more advanced
features that may be needed when developing ``real'' parallel
applications.

% PM2 Step by Step
%_____________________________________________________________________________
\section{Hello World!}

Let us start our \pmm tour by writting the traditional ``\emph{Hello
  World!}'' program. Then, we will extend it step by step to cover the
main functionalities provided by the \pmm programming interface.

\subsection{The minimal PM2 program}

\begin{figure}%[p]
\begin{center}
\begin{boxedminipage}{0.8\textwidth}
\begin{footnotesize}
\begin{listing}{0}
 #include <stdio.h>
 #include <pm2.h>

 int pm2_main(int argc, char *argv[])
 {
   pm2_init(&argc, argv);

   printf("Hello World!\n");

   if(pm2_self() == 0)
     pm2_halt();

   pm2_exit();
   return 0;
 }
\end{listing}
\end{footnotesize}
\end{boxedminipage}
\end{center}
\caption{Minimal \pmm program\label{fig:ex1}}
\end{figure}
%\afterpage{\clearpage}

The Figure~\ref{fig:ex1} (page~\pageref{fig:ex1}) shows an example of
a minimal \pmm code. Surprising though, this is a parallel program
that will spawn on several nodes during execution! A \pmm application
always runs the same code on each node. However, this execution is
asynchronous most of the time. In this respect, \pmm programs obey an
\emph{SPMD} programing style. So remember that our example program
will be executed on several nodes concurrently. 

Let us now concentrate on the minimal rules that a \pmm program must
obey...

\subsubsection{Header files}

As we can see (line~$1$), a \pmm program must always include the
\pmm-specific \texttt{pm2.h} header file, along with other standard
header files.  Note that this is the \emph{only} \pmm header file that
has to be included by user applications.

\subsubsection{The \texttt{main} function}

Line~$5$ shows that the main function of the \pmm program is named
\verb+pm2_main+, contrary to traditional {\sc C} programs that use the
well-known \verb+main+ function name. The arguments remain the regular
\texttt{argc}/\texttt{argv} pair.

The reason is that the \pmm library actually defines its own
\verb+main+ function that calls the \verb+pm2_main+ one after some
internal stack manipulation. This allows \pmm to greatly enhance the
performance of various thread management functions. Although most
programs (or programers) will accomodate such a ``C convention
violation'', there are some applications that may require to use the
regular \verb+main+ function name. In particular, it can be the case
with applications linked with non-C code (\emph{e.g.}, Fortran code).
In this case, please refer to Section \ref{sec:tradimain}
(page~\pageref{sec:tradimain}).

\subsubsection{\pmm initialization}

A \pmm program has to call \verb+pm2_init+ to effectively initialize
the \pmm runtime system. Moreover, this step involves a global
synchronization scheme (actually a barrier) between each \pmm process
so that, when returning from \verb+pm2_init+, each process is assigned
a unique \emph{rank number} (see Section \ref{sec:rank},
page~\pageref{sec:rank}).

Most importantly, the \verb+pm2_init+ function spawns a number of
internal \emph{thread daemons} that are in charge of listening the
network and answering to external requests (such as RPC request,
incoming thread migrations, etc.) That means that, in some programs,
several application threads may run concurrently to the main program
at this point. If some initialization code needs to be performed
before any thread is started, then this code must be called
\emph{before} \verb+pm2_init+. See Section \ref{sec:startupfunc}
(page~\pageref{sec:startupfunc}) for a more complete discussion about
this initialization topic.

Note that the Unix standard input/output streams may not be
initialized correctly before the call to \verb+pm2_init+. Thus, the
behavior of programs using I/O operations before \verb+pm2_init+ is
not defined.

Going back to our example program, we now observe (line~$7$) a call to
the \verb+printf+ C library function. As we mentioned previously, all
processes will perform this call and will produce the ``\emph{Hello
  World!}'' string on their output stream. Section \ref{sec:output}
(page~\pageref{sec:output}) describes where these outputs
will actually go during execution.

\subsubsection{Who's who?}
\label{sec:rank}

As mentioned previously, every process belonging to the same \pmm
``session'' has its own unique \emph{rank} number. Such a value is
comprised between $0$ and \verb+pm2_config_size()+$-1$. A process can
get its rank by calling \verb+pm2_self()+.

The process with the $0$ rank number has a particular status because
it is the only one which whose input/output streams are linked to the
terminal the application was launched from. We later refer to this
process as the ``\emph{main process}'' of an application. As a
consequence, only the main process of an application may access its
standard input stream (\emph{e.g.}, using \verb+scanf+, \verb+fgets+,
etc.)

As an example of use of \verb+pm2_self()+, line~$9$ features a test on
the rank of the current process: if the current process is the main
process, then \verb+pm2_halt()+ is called.

\subsubsection{Session termination}

The termination of a \pmm program is a delicate task since \pmm
performs no automatic termination detection. Thus, the termination
decision must be made at the application level. The termination phase
is splitted in two steps.

Exactly one process must perform an explicit call to
\verb+pm2_halt()+. In our example, the main process was chosen to
execute this function. Actually, any process could have been
selected. For instance, we could replace line~$9$ by:
\begin{small}
\begin{verbatim}
    if(pm2_self() == pm2_config_size()-1)
\end{verbatim}
\end{small}

At least conceptually, \verb+pm2_halt+ performs a broadcast that asks
all processes to stop answering to requests from the outside world. In
fact, this steps cuts the links between processes.  Actually, it stops
the internal daemon threads that are in charge of polling the network.
Note that after a process receives this order, it still continues its
execution (it is not ``killed'') in a \emph{standalone} mode.

To exit from a \pmm session, each process must call \verb+pm2_exit()+.
This call blocks the calling thread until all other threads (belonging
to the same process) terminate. In our example, no other application
thread but the main one are running. Thus, \verb+pm2_exit+ will
complete as soon as the internal daemon threads are stopped. If we
would remove the call to \verb+pm2_halt+, then every process would
execute for an infinite time waiting for external requests...

From the user point of vue, the \pmm program terminates as soon as the
shell prompts for the next command within the terminal. This
corresponds to the end of the main process. However, note that some
processes may actually still run after the main process has completed.
In our example, the main process can terminate even before any other
process begun to execute \verb+printf+! This is not a problem however,
since \verb+printf+ does not involve any communication with other
processes. We will further discuss this termination problem in Section
\ref{sec:output} (page~\pageref{sec:output}).

\subsection{Running a \pmm program}

Lets us now see how to configure, compile and run our \pmm program on
a parallel architecture.

\subsubsection{Configuring and compiling the program}

As we described in earlier sections, one may first configure the
target network protocol before compiling an application. For instance,
if we assume that the MPI-LAM communication library is available on
our target machine, then we shall type:

\begin{small}
\begin{verbatim}
        > pm2custom mpi-lam
        The current network interface is set to: mpi-lam
\end{verbatim}
\end{small}

The \pmm library is now correctly configured and we can compile our
example program on the yarget machine. Since the ``\emph{Hello
  World!}'' program is part of the \pmm original distribution, we just
have to go to the right directory and launch the compiler:

\begin{small}
\begin{verbatim}
        > cd ${PM2_ROOT}/examples/simple
        > make hello
\end{verbatim}%$
\end{small}

The compiled version of our program is automatically placed in a
directory whose name is architecture dependent: \verb+LINUX/X86+ on
Intel Linux machines, \verb+SOLARIS/SPARC+ on Sun workstations, etc.

\subsubsection{Session topology definition}

The last step before execution is to specify the list of hostnames on
which the \pmm application is going to run. For this, one must use the
\texttt{pm2conf} command. Example for a session using the first three
nodes on a cluster called \texttt{POPC}:
\begin{small}
\begin{verbatim}
    > pm2conf popc1 popc2 popc3
    The current PM2 configuration contains 4 host(s):
    0 : popc1
    1 : popc2
    2 : popc3
\end{verbatim}
\end{small}

In some situations, it may be useful to use a given node name several
times. For instance, here is how to configure a two-processes \pmm
execution on your local machine:
\begin{small}
\begin{verbatim}
    > pm2conf localhost localhost
    The current PM2 configuration contains 2 host(s):
    0 : localhost
    1 : localhost
\end{verbatim}
\end{small}

See section \ref{sec:scripts} (page~\pageref{sec:scripts}) for an
exhaustive description of \verb+pm2conf+ options.

\subsubsection{Launching the program}

Although it is not a requirement in general, some low-level network
libraries require the use of specific startup scripts. That for, the
\texttt{pm2load} shell script takes care about all the details of
setting up the \pmm session, whatever the underlying network protocol
is. This script takes the application program name and its arguments
as parameters.

In our case, let us assume that the compiled version of our sample
\pmm program is located in \verb+./LINUX/X86+. We can launch it by
simply typing the following command:
\begin{small}
\begin{verbatim}
    > pm2load ./LINUX/X86/hello
    Hello World !
    [Threads : 4 created, 0 imported (0 cached)]
\end{verbatim}
\end{small}

Congratulations, you've run your first \pmm program! \verb+;-)+

We can see that our program generates two messages. The first one is
the expected \emph{Hello World!} message coming from the master
process \texttt{printf}. The second one is some \pmm running
statistics related to the master process.

Note that the output of the other two processes (assuming an execution
on \verb+popc1+, \verb+popc2+ and \verb+popc3+) do not appear in
the terminal.

\subsubsection{Problems?}

Feedback from \pmm users showed that several common pitfalls may be
encountered at this time. If this is your case, please refer to
Section \ref{sec:commonpitfalls}: ``\emph{Common Pitfalls}''.

\subsection{Where does the output go?}
\label{sec:output}

One can notice that the previous example only printed the message
generated by the master node process. Indeed, the standard output of
the other processes is redirected to log files located into the /tmp
directory of each node. The logs are easily accessible using the
command \texttt{pm2logs} which is charged to retrieve and display logs
from each slave node of the session configuration.

\begin{figure}%[p]
\begin{center}
\begin{boxedminipage}{0.8\textwidth}
\begin{footnotesize}
\begin{listing}{0}
 #include <stdio.h>
 #include <pm2.h>

 int pm2_main(int argc, char *argv[])
 {
   pm2_init(&argc, argv);

   if (pm2_self() == 1)
     {
       printf("Hello World !\n");    
       pm2_halt();
     }
  
   pm2_exit();
   return 0;
 }
\end{listing}
\end{footnotesize}
\end{boxedminipage}
\end{center}
\caption{Output redirection, example 1\label{fig:ex2-1}}
\end{figure}

\subsubsection{Redirecting output to the terminal}

\begin{figure}%[p]
\begin{center}
\begin{boxedminipage}{0.8\textwidth}
\begin{footnotesize}
\begin{listing}{0}
 #include <stdio.h>
 #include <pm2.h>

 int pm2_main(int argc, char *argv[])
 {
   pm2_init(&argc, argv);

   if (pm2_self() == 1)
     {
       pm2_printf("Hello World !\n");    
       pm2_halt();
     }
  
   pm2_exit();
   return 0;
 }
\end{listing}
\end{footnotesize}
\end{boxedminipage}
\end{center}
\caption{Output redirection, example 2\label{fig:ex2-2}}
\end{figure}
%\afterpage{\clearpage}

\section{Remote Procedure Calls}
%===========
\subsection{Invoking a remote service}

\begin{figure}%[p]
\begin{center}
\begin{boxedminipage}{0.8\textwidth}
\begin{footnotesize}
\begin{listing}{0}
 #include <stdio.h>
 #include <pm2.h>

 static unsigned int service_id;

 static void service(void)
 {
   pm2_rawrpc_waitdata();
   printf("Hello, World!\n");
 }

 int pm2_main(int argc, char *argv[])
 {
   pm2_rawrpc_register(&service_id, service)
   pm2_init(&argc, argv);

   if (pm2_self() == 0)
     {
       pm2_rawrpc_begin(1, service_id, NULL);
       pm2_rawrpc_end();

       pm2_halt();
     }
   pm2_exit();
   return 0;
 }
\end{listing}
\end{footnotesize}
\end{boxedminipage}
\end{center}
\caption{Sample service\label{fig:ex3}}
\end{figure}
%\afterpage{\clearpage}

%===========
\subsection{Passing parameters}

\begin{figure}%[p]
\begin{center}
\begin{boxedminipage}{0.9\textwidth}
\begin{footnotesize}
\begin{listing}{0}
#include <stdio.h>
#include <pm2.h>

static unsigned int service_id;

static void service(void)
{
  int len;
  char *s;

  pm2_unpack_int(SEND_CHEAPER, RECV_EXPRESS, &len, 1);
  s = malloc(len);
  pm2_unpack_byte(SEND_CHEAPER, RECV_CHEAPER, s, len);
  pm2_rawrpc_waitdata();

  printf("The sentence is %s\n", s);
}

int pm2_main(int argc, char *argv[])
{
  pm2_rawrpc_register(&service_id, service)
  pm2_init(&argc, argv);
  if (pm2_self() == 0)
    {
      char s[] = "A la recherche du temps perdu.";
      int len;

      len = strlen(s) + 1;
      pm2_rawrpc_begin(1, service_id, NULL);
      pm2_pack_int(SEND_CHEAPER, RECV_EXPRESS, &len, 1);
      pm2_pack_byte(SEND_CHEAPER, RECV_CHEAPER, s, len);
      pm2_rawrpc_end();

      pm2_halt();
    }
  pm2_exit();
  return 0;
}
\end{listing}
\end{footnotesize}
\end{boxedminipage}
\end{center}
\caption{Service with a string parameter\label{fig:ex4}}
\end{figure}
%\afterpage{\clearpage}

\subsubsection{Packing/unpacking API}

A messages consists of several pieces of data, located anywhere in
userspace. They are constructed (resp. re-constructed) incrementaly
using {\em packing} (resp. {\em unpacking}) primitives, possibly at
multiple software levels without losing efficiency. The following
example illustrates this need. Let us consider a remote procedure call
which takes an array of unpredictable size as a parameter. When the
request reaches the destination node, the header is examined both by
the multithreaded runtime (to allocate the appropriate thread stack
and then to spawn the server thread) and by the user application (to
allocate the memory where the array should be stored).

The critical point of a send operation is obviously the series of {\em
  packing} calls. Such packing operations simply {\em virtually}
append the piece of data to a message under construction. In addition
to the address of data and its size, the packing primitive features a
\emph{flag} parameter which specifies the semantics of the operation.
Available send flags are defined as follows:
\begin{description}
    
  \item[\tt SEND\_SAFER] This flag indicates that \pmm should pack
    the data in a way that further modifications to the corresponding
    memory area should not corrupt the message. This is particularly
    mandatory if the data location is reused before the message is
    actually sent.

  \item[\tt SEND\_LATER] This flag indicates that \pmm should not
    consider accessing the value of the corresponding data until the
    \verb+mad_end_packing+ primitive is called. This means that any
    modification of these data between their packing and their sending
    shall actually update the message contents.
    
  \item[\tt SEND\_CHEAPER] This is the default flag. It allows \pmm
    to do its best to handle the data as efficiently as possible. The
    counterpart is that no assumption should be made about the way
    \pmm will access the data. Thus, the corresponding data should
    be left unchanged until the send operation has completed. Note
    that most data transmissions involved in parallel applications can
    accomodate the \verb+send_CHEAPER+ semantics.

\end{description}
The following flags control the reception of user data packets:
\begin{description}
    
  \item[\tt RECV\_EXPRESS] This flag forces \pmm to guarantee
    that the corresponding data are immediately available after the
    the {\em unpacking} operation. Typically, this flag is mandatory
    the data is needed to determine the next forthcoming
    \emph{unpacking} calls. On some network protocols, this
    functionality may be available for free. On some others, it could
    penalize latency and bandwidth. The user should therefore extract
    data this way only when necessary.
    
  \item[\tt RECV\_CHEAPER] This flag allows \pmm to defer the
    extraction of the corresponding data until the execution of
    \verb+mad_end_unpacking+. Thus, no assumption can be made about
    the exact moment at which the data will be extracted. Depending on
    the underlying network protocol, \pmm will do its best to
    minimize the overall message transmission time. If combined with
    \verb+send_CHEAPER+, this flag always guarantees that the
    corresponding data is transmitted as efficiently as possible.

\end{description}

\subsubsection{A more interesting example}

%===========
\subsection{"Threaded" services}
        Services handlers are executed sequentially
        Thread creation: pm2\_thread\_create
        Caveat! pm2\_rawrpc\_waitdata...
\endchap

%%%%%%%%%%%%%%%%%%%%%%%%%%%%%%%%%%%%%%%%%%%%%%%%%%%%%%%%%%%%%%%%%%%%%%%%%%%%%%
\chapter{Mastering \pmm}

\section{More about Remote Procedure Calls}

\subsection{Invoking services synchronously}
        2 one-way invocations + synchronization

\subsubsection{Examples}
        Better: using the "pm2\_rawrpc\_wait" features

% Thread migration
%_____________________________________________________________________________
\section{Thread migration}
        Thread state : migratable/non-...

\subsection{Migration in 3 steps}
        freezing the local scheduler
        getting the list of migratable threads
        selecting and migrating threads

\subsection{Example}
        Iso-address migration

\subsection{Migrating with dynamically allocated data}

\subsection{Isomalloc}

% DSM
%_____________________________________________________________________________
\section{Sharing data across clusters}
        DSM-PM2

% Advanced Threading
%_____________________________________________________________________________
\section{Advanced thread features}

\subsection{Thread synchronization}

% Miscellaneous
%_____________________________________________________________________________
\section{Miscellaneous}

\subsection{Application initialization with \pmm}
\label{sec:startupfunc}

\verb+pm2_set_startup_func+

\endchap

%%%%%%%%%%%%%%%%%%%%%%%%%%%%%%%%%%%%%%%%%%%%%%%%%%%%%%%%%%%%%%%%%%%%%%%%%%%%%%
\chapter{Additional \pmm material}

% Standard Main
%_____________________________________________________________________________
\section{Using a standard main}
\label{sec:tradimain}

Some application may require to use the traditional \verb+main+
function name instead of \verb+pm2_main+. In this case, the
\verb+STANDARD_MAIN+ compilation flag must be defined for each \pmm
file. This can be done by adding the definition in the \pmm
makefile (\verb+pm2/make/common.mak+):

\begin{small}
\begin{verbatim}
   GLOBAL_OPTIONS   =   -Wall -O6 -DSTANDARD_MAIN
\end{verbatim}
\end{small}

Note that this may affect the performance of several thread management
functions...

% Scripts
%_____________________________________________________________________________
\section{Synopsys of PM2 scripts}
\label{sec:scripts}

\subsection{pm2custom}

\subsection{pm2conf}

\subsection{pm2load}

% Common Pitfalls
%_____________________________________________________________________________
\section{Common Pitfalls}
\label{sec:commonpitfalls}

The Table~\ref{tbl:pitfalls} sums up the
symptoms together with possible solutions.
\begin{table}
\caption{Common Pitfalls\label{tbl:pitfalls}}
\begin{center}
\begin{tabular}{|p{5cm}|p{7cm}|}                             \hline
Symptoms                       & Solutions                    \\ \hline
`Permission denied' error      & Each configuration node must be made
accessible by \texttt{rsh} from the local host: update you
\texttt{.rhosts} file                                         \\ \hline
\end{tabular}
\end{center}
\end{table}

% FAQ
%_____________________________________________________________________________
\section{FAQ}
\label{sec:faq}
\begin{quote}
    Why do I get the message "Permission denied" when running pm2load?

    Why does my program crash the entire cluster?
\end{quote}

% Bibliography
%_____________________________________________________________________________
\section{Bibliography}
    
\endchap
            
\end{document}
_______________________________________________________________________________
$Log: getting_started.tex,v $
Revision 1.15  2000/02/14 09:51:11  rnamyst
Many modifications... The structure has also been modified a little.

Revision 1.14  2000/02/05 13:17:53  oaumage
- getting_started.tex: ajout du tableau `pitfalls'

Revision 1.13  2000/01/26 17:38:32  oaumage
- quelques modifs ...

Revision 1.12  2000/01/26 11:30:53  oaumage
- exemple 1 complete
- diverses corrections et modifications

Revision 1.11  2000/01/25 17:11:31  oaumage
- diverses modifications et corrections

Revision 1.10  2000/01/25 10:15:50  oaumage
- passage en mode Report: il semble que se soit plus facile a lire, et il
  s'agit du style generalement employe pour une documentation

Revision 1.9  2000/01/25 10:03:06  oaumage
- remplissage de la premiere partie (intro/installation)
- un petit clin d'oeil pour la page de titre

Revision 1.8  2000/01/21 17:26:00  oaumage
- modification des exemples pour tenir compte de l'evolution des fonctions
  pack/unpack

Revision 1.7  2000/01/17 13:36:00  oaumage
- correction des exemples 2-1 et 2-2

Revision 1.6  2000/01/14 17:16:00  oaumage
- ajout des exemples 2, 3 et 4

Revision 1.5  2000/01/14 16:06:15  oaumage
- exemple minimal + texte

Revision 1.4  2000/01/14 15:38:05  oaumage
- nouveau programme minimal

Revision 1.3  2000/01/14 14:05:45  oaumage
- quelques corrections au niveau du titre
- ajout de l'arborescence et de la disponibilite

Revision 1.2  2000/01/14 13:25:34  oaumage
- quelques corrections pour que le document compile

Revision 1.1  2000/01/14 13:10:54  oaumage
- Document Latex `Getting started with PM2'
