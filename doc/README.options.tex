\documentclass[11pt, a4paper ,twoside]{article}
\usepackage{epstopdf}
\usepackage{natbib}
\usepackage{epstopdf}
\usepackage[utf8]{inputenc}
\usepackage{a4wide}
\usepackage{rotating}
\usepackage{multicol}
\usepackage{afterpage}
\usepackage{color}
\usepackage{array}
\usepackage{floatflt}
\usepackage[colorlinks=true, linkcolor=blue, bookmarks=true, citecolor=blue, pagebackref]{hyperref}

\setcounter{secnumdepth}{3}
\setcounter{tocdepth}{1}
\usepackage[french]{minitoc}
\setcounter{parttocdepth}{2}
\usepackage{amsmath}

\usepackage[final]{pdfpages}
\usepackage{amssymb}
\usepackage[english, french]{babel}
\pagestyle{plain}

\begin{document}

\title{Quick Guide on Flavor Configurations}
\author{The PM2 Team}
\date{}
\maketitle

\section{Introduction}

There are two kinds of options for each module:
\begin{itemize}
\item inclusive (check boxes),
\item exclusives (radio boxes), only one choice per section.
\end{itemize}

For each of these options, there can be a text box that allows to
specify a value for that option, and even provide a default value.
The only condition for this is to append ``:'' to the name of that
option.

There is for instance: \verb|--marcel-stacksize:64| (64 being the
default value).

The following files should be put in \verb|module-name/config/options| to
define an option:
\begin{enumerate}
\item \verb|nnnoption-name:.sh| action of the option
\item \verb|nnnoption-name:.help| provides help on the option
\item \verb|nnnoption-name:.dep| specifies eventual dependencies
\item \verb|nnnoption-name:.dft| specifies a default value
\end{enumerate}

\verb+nnn+ is the number of the option and is used for sorting them. It
should start with a 0 for inclusive options, and 1 for exclusive
options.

Help files just contain the text that should be disabled \emph{e.g.} in
popups in \verb+ezflavor+.

After adding/supressing/modifying options, \verb+pm2-recreate-links+ has
to be called to take them into account.

In addition to options, \verb+module-name/config/module-name.sh+
is included before options, and
\verb+module-name/config/module-name-post.sh+ is included after options,
\verb+module-name/config/module-name.dep+ specify the dependencies for
the module itself, and \verb+module-name/config/module-name.sh+ provides
help on the module itself.

\section{Actions}

A flavor is just a concatenation of all the \verb+.sh+ files of all the
options that have been selected.  These are shell script snippets that
can do anything they need.  Their eventual effect is to modify some
environment variables.  In the lists below, \verb+${MOD}+ is the module
name in capital letters.  The \verb+appli+ module corresponds to 
the application itself.

Read-only variables
\begin{itemize}
\item \verb+PM2_ARCH+ is set to the target architecture. See
\verb+bin/pm2-arch+ for a list.
\item \verb+PM2_SYS+ is set to the target architecture. See
\verb+bin/pm2-sys+ for a list.
\end{itemize}

\begin{itemize}
\item \verb+PM2_${MOD}_BUILD_DYNAMIC+ is set to \verb+yes+ by the
\verb+build_dynamic+ option to request compiling the module as a shared
library.
\item \verb+PM2_${MOD}_BUILD_STATIC+ is set to \verb+yes+ by the
\verb+build_dynamic+ option to request compiling the module as a static
library.
\item \verb+PM2_${MOD}_CFLAGS+ contains the C flags that will be used
while compiling all modules.
\item \verb+PM2_${MOD}_CFLAGS_KERNEL+ contains the C flags that will be
used while compiling module \verb+${MOD}+.
\item \verb+PM2_${MOD}_LD_PATH+ contains \verb+-L+ options that will be
passed to the linker when linking a shared version of the module, or
when statically linking the module into an application.
\item \verb+PM2_${MOD}_LIBS+ contains \verb+-l+ options that will be
passed to the linker when linking a shared version of the module, or
when statically linking the module into an application.
\item \verb+PM2_${MOD}_EARLY_LDFLAGS+ contains linker flags that will be
passed before objects.
\item \verb+PM2_${MOD}_EARLY_LDFLAGS_KERNEL+ contains linker flags that
will be passed before objects, but only when linking the module itself.
\item \verb+PM2_${MOD}_DYN_LIBS_DIR+ contains a list of directories
which will be added to \verb+LD_LIBRARY_PATH+ by \verb+pm2-load+.
\item \verb+PM2_${MOD}_MAKEFILE+ contains text that will be pasted as is
in the module's \verb+Makefile+.
\item \verb+PM2_${MOD}_LIBNAME+ overrides the name of the produced
library (which is the name of the module by default).
\item \verb+PM2_${MOD}_LINK_OTHERS+ should be set to \verb+yes+ if the
module needs to be linked against all other modules, so that the
module's shared library will pull all the others at runtime.  This is
notably needed when building ABI-compatibility libraries like
\verb+libpthread.so+.
\item \verb+PM2_COMMON_PM2_SHLIBS+ contains the list of modules that are
compiled as shared libraries.
\item \verb+PM2_COMMON_FORTRAN_TARGET+ specifies which fortran compiler
should be used, if any.
\item \verb+PM2_CC+ specifics which C compiler should be used.
\item \verb+PM2_PROTOCOLS+ contains the list of Madeleine protocols that
will be available.
\item \verb+PM2_NMAD_DRIVERS+ contains the list of Madeleine drivers.
\item \verb+PM2_NMAD_INTERFACES+ contains the list of Madeleine
interfaces.
\item \verb+PM2_NMAD_LAUNCHER+ specifies which launder Madeleine should
use.
\item \verb+PM2_NMAD_STRATEGIES+ contains the list of Madeleine
strategies.
\item \verb+PM2_NMAD_STRAT+ specifies which Madeleine strategy should
be used.
\item \verb+PM2_LOADER+ specifies which loader should be run by
\verb+pm2-load+.
\end{itemize}

The following variables don't seem to have effect any more.
\begin{itemize}
\item \verb+PM2_${MOD}_MODULE_DEPEND_LIB+ was probably used for
inter-module dependencies.
\end{itemize}

The following functions are provided (see \verb+bin/pm2-config-tools.sh+).
\begin{itemize}
\item \verb+included_in what where+ looks for the word \verb+what+ in
the string \verb+where+.
\item \verb+not_included_in+ does the contrary.
\item \verb+defined_in what where+ looks for the word \verb+-Dwhat+ in
the string \verb+where+.
\item \verb+not_defined_in+ does the contrary.
\end{itemize}


\section{Dependencies}

Some options require that others be set.  For instance, module debug
support depends on the debug part of TBX.  This is expressed in
the \verb+.dep+ files: \verb+tbx/config/options/00debug.dep+ has a
\verb+Provide: TBX_debug+ line, and \verb+*/config/options/00debug.dep+
contain a \verb+Depend: TBX_debug+ line. The name of the dependency,
\verb+TBX_debug+, but should be prefixed by the name of the module that
provides it. Conflicts can also be expressed by putting a
\verb+Conflict:+ line in the \verb+.dep+ file. The special case of
depending on a module itself is handled by just using
\verb+Depend: name+.

\section{Generic options}

A few options are common to all modules: debugging flags, linking
options, etc. \verb+generic/config/options+ contains templates for
these.  Modifications thus need to be done there, and
\verb+make duplique_global+ run from there to propagate the modification
to all modules.

\section{Common options}

A few options are not specific to any modules, they live in
\verb+common/config/options+.

\section{Handling flavors}

To configure flavors, either use \verb+ezflavor+, \verb+make config+, or
use the following commands:
\begin{itemize}
\item \verb|pm2-flavor get --flavor=flavor-name| : Shows the
flavor.
\item \verb|pm2-flavor list| : Give the list of flavors.
\item \verb|pm2-flavor set --flavor=flavor-name| : Reset the whole
flavor.
\item \verb|pm2-flavor set --flavor=flavor-name `cat FILE`| : Set
flavor options from a file.
\item \verb|xargs pm2-flavor set --flavor=flavor-name < FILE| : Other
way to achieve the same.
\item \verb|pm2-create-sample-flavor flavor-name| : Generate one of the
sample flavors.
\end{itemize}
\end{document}
